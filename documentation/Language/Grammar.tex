\documentclass[main.tex]{subfiles}
\begin{document}

    \chapter{Grammar}
    There are two types of top level definitions:
    \begin{itemize}
        \item \textbf{structs}
        \item \textbf{functions}
    \end{itemize}
    \section{Structs}
        Structs start with the keyword \textit{struct} and are followed by an identifier.
        Inside of curly braces, you can define member variables as follows:
        \textit{name} : \textit{type} ;
        Default values are not supported yet.
        If the value is to be changed later, add the mut keyword.
        mut \textit{name} : \textit{type};
    \section{Functions}
    The definition of a function looks like this:
    func \textit{name} (\textit{parameters}) -> \textit{return type} statement
    statement may be a single statement or a block of statements.
    The return type may be omitted. In that case it is void.
    Parameters are in the same format as struct members, but separated by commas.
    e.g. func foo(x:int,mut y:float)\{...\}
    
\end{document}